\documentclass[11pt,letterpaper,sans]{moderncv}   

\usepackage{graphicx} 
\usepackage{longtable}
\usepackage{multirow}
\usepackage{textcomp}
\usepackage{units}
\usepackage{lineno}
\usepackage{rotating}
\usepackage{amssymb}
\usepackage{amsmath}
\usepackage[utf8]{inputenc}
\usepackage{longtable}
\usepackage{comment}
\usepackage{xcolor}

\moderncvstyle{banking}
\moderncvcolor{blue}
\definecolor{color1}{rgb}{0.0, 0.45, 0.81}
\renewcommand{\familydefault}{\sfdefault}\usepackage[top=2cm,bottom=2cm,left=2cm,right=2cm,bindingoffset=0cm]{geometry}
\setlength{\hintscolumnwidth}{3cm}
\usepackage{enumitem} 
\setlist{nolistsep}

\firstname{Davide}
\familyname{Gerosa}
\title{Curriculum Vitae}
\phone{+1~626-395-6829}
\email{dgerosa@caltech  $\;\;\bullet\;\;$ \today}
%\email{dgerosa@caltech} 

%\extrainfo{\today}
%{ } 
%\homepage{www.tapir.caltech.edu/\textasciitilde dgerosa}       
%\photo[70pt][0.1pt]{myphoto}                  
% '64pt' is the height the picture must be resized to, 0.4pt is the thickness of the frame around it (put it to 0pt for no frame) and 'picture' is the name of the picture file; optional, remove the line if not wanted
%\quote{\small {I am relativistic astrophysicist, studying the impact of Einstein's general relativity on the astrophysical world.
%My research interests span from binary black-holes as gravitational-wave sources, to gas accretion onto black-holes, galaxy modelling and alternative theories of gravity.}}         
% to show numerical labels in the bibliography (default is to show no labels); only useful if you make citations in your resume
\makeatletter
\renewcommand*{\bibliographyitemlabel}{\@biblabel{\arabic{enumiv}}}
\makeatother

\newcommand{\mnras}{Monthly Notices of the Royal Astronomical Society}
\newcommand{\mnrasl}{Monthly Notices of the Royal Astronomical Society Letters}
\newcommand{\prd}{Physical Review D}
\newcommand{\prl}{Physical Review Letters}
\newcommand{\cqg}{Classical and Quantum Gravity}


\long\def\suppress#1\endsuppress{%
  \begingroup%
    \tracinglostchars=0%
    \let\selectfont=\nullfont
    \nullfont #1\endgroup}


\begin{document}

\makecvtitle

%\cvitem{}{\emph{\vspace{-1cm}\\
%$\quad$ I am a relativistic astrophysicist, studying the impact of Einstein's general relativity on the astrophysical world.
%My research interests span from binary black holes as gravitational-wave sources, to gas accretion onto black holes, black-hole recoils, and alternative theories of gravity.}         
%}

\cvitem{}{\emph{\vspace{-1cm}\\
$\quad$ Relativistic astrophysicist and gravitational-wave astronomer, studying the impact of Einstein's general relativity on the astrophysical world. Research interests include astrophysical inference with gravitational-wave sources, black-hole binary spin dynamics, black-hole recoils, accretion disks and alternative theories of gravity.}}


\section{Personal information}
%\cvitem{Present position}{Ph.D. candidate.  \emph{University of Cambridge},   
%\newline{}
%\protect{$~\qquad\qquad\qquad\quad\;\;\;$}
%Department of Applied Mathematics and Theoretical Physics (DAMTP), 
%\newline{}
%\protect{$~\qquad\qquad\qquad\quad\;\;\;$}
%Centre for Mathematical Sciences, Wilberforce Road, Cambridge CB3 0WA, UK.}  
\cvitem{Present position}{NASA Einstein Fellow.
\newline{} 
\protect{$~\qquad\qquad\qquad\quad\;\;\;$}
\emph{California Institute of Technology},  TAPIR MC 350-17,
\newline{}
\protect{$~\qquad\qquad\qquad\quad\;\;\;$}
1200 E California Blvd, Pasadena, CA 91125, USA.}  
\cvitem{Personal webpage}{\href{http://www.tapir.caltech.edu/~dgerosa}{\texttt{www.tapir.caltech.edu/\textasciitilde dgerosa}}}
%\cvitem{Date of birth}{May 26, 1988.}
%\cvitem{Citizenship}{Italian, EU.}
%\cvitem{Place of birth}{Sesto San Giovanni MI, Italy.}
%\cvitemwithcomment{English}{Fluent}{TOEFL exam:  109/120, Oct.2012}

%\section{References}
%\begin{longtable}{lp{0.3cm}p{16cm}}
%\textbf{Ulrich Sperhake} && Lecturer; University of Cambridge, Cambridge, UK.  \\
% && Visiting Associate; Caltech, Pasadena CA, USA.  \\
% && \texttt{u.sperhake@damtp.cam.ac.uk} 
%  \vspace{0.2cm} \\
%\textbf{Emanuele Berti} && Associate Professor; University of Mississippi, Oxford MS, USA.  \\
% && Research Professor; Instituto Superior Tecnico, Lisbon, Portugal.  \\
% && \texttt{eberti@olemiss.edu}
% \vspace{0.2cm} \\
%\textbf{Michael Kesden} && Assistant Professor; University of Texas at Dallas, Richardson TX, USA. \\
% && \texttt{kesden@utdallas.edu}
%\end{longtable}

\section{Research experience}

\cventry{2016 - {current}}{NASA Einstein Fellow}{California Institute of Technology}{Pasadena CA, USA}{}{}
\vspace{-0.1cm}
\begin{tabular}{rcl}
&\hspace{0.4cm} &$\circ\;\;${\textit{Support}}: Einstein postdoctoral prize fellowship.  Part of the NASA prize fellowship program, Einstein\\
&\hspace{0.4cm} &  \hspace{0.4cm}Fellowships are prestigious awards in theoretical astrophysics consisting of personal research grant at\\
&\hspace{0.4cm} &  \hspace{0.4cm}selected US institution. See  \href{http://cxc.cfa.harvard.edu/fellows}{\texttt{cxc.cfa.harvard.edu/fellows}}.\\
\end{tabular}
\vspace{0.3cm}


\cventry{2013 - 2016}{Ph.D. candidate}{University of Cambridge}{Cambridge, UK}{}{}
\vspace{-0.1cm}
\begin{tabular}{rcl}
&\hspace{0.4cm} &$\circ\;\;${\textit{Support}}: Isaac Newton Studentship; STFC Ph.D. Studentship. The 
Isaac Newton Studentship is the \\&\hspace{0.4cm} &  \hspace{0.4cm}
most prestigious prize PhD scholarship in astronomy of the University of Cambridge.\\
&\hspace{0.4cm} &$\circ\;\;${\textit{Supervisor}}: Ulrich Sperhake.\\
&\hspace{0.4cm} &$\circ\;\;${\textit{Thesis}}: Source modelling at the dawn of gravitational-wave astronomy. Resulted in 14 publications.\\
%&\hspace{0.4cm} &$\circ\;\;${\textit{Teaching assistant}}: Introduction to General Relativity (Part II, 3rd year undergraduate class);\\
%&\hspace{0.4cm} &\hspace{3.5cm} Advanced General Relativity (Part III, master class).\\
%&\hspace{0.4cm} &$\circ\;\;${\textit{Undergraduate summer project mentoring}}: J.~Vosmera, University of Cambridge, 2015;\\
%&\hspace{0.4cm} &\hspace{7.3cm} R.~Barbieri, University of Pavia, 2016.\\
\end{tabular}


%\vspace{0.3cm}
%\cventry{Jun-Aug 2012}{LIGO Summer Undergraduate Research Fellow}{California Institute of Technology}{Pasadena CA, USA}{}{}
%%\cvitem{}{Scientific research within the LIGO SURF (Summer Undergraduate Research Fellowship) program. %Project developed as part of the LIGO Scientific Collaboration.}
%%\cvitem{}{Project title: Spin Alignment Effects in Black Hole Binaries.} 
%%\cvitem{}{Advisor: Emanuele Berti.} 
%\vspace{-0.1cm}
%\begin{tabular}{rcl}
%&\hspace{0.4cm} &$\circ\;\;${\textit{Advisor}}:  Emanuele Berti.\\
%\end{tabular}

%\vspace{0.3cm}
%\textbf{\textcolor{color1}{Extended research visits:}}\vspace{0.15cm}\\ 
%\begin{tabular}{rcl}
%&\hspace{0.4cm} &$\circ\;\;${\textbf{Institut Astrophysique de Paris}}. Paris, France. \textit{Jul 2015}\\
%&\hspace{0.4cm} &$\circ\;\;${\textbf{University of Mississippi}}. Oxford MS, USA. \textit{Aug-Dec 2012}\\
%&\hspace{0.4cm} &$\circ\;\;${\textbf{Caltech}}. Pasadena CA, USA. LIGO Summer Undergraduate Research Fellow. \textit{Jun-Aug 2012}\\
%\end{tabular}


\vspace{0.3cm}
\textbf{\textcolor{black}{Extended research visits:}}\vspace{0.05cm}\\ 
\cvitemwithcomment{}{\hspace{0.4cm}$\circ\;$ Institut Astrophysique de Paris. Paris, France.}{Jul 2015} \vspace{-0.1cm}
\cvitemwithcomment{}{\hspace{0.4cm}$\circ\;$ University of Mississippi, Oxford MS, USA.}{Aug-Dec 2012} \vspace{-0.1cm}
\cvitemwithcomment{}{\hspace{0.4cm}$\circ\;$ Caltech, Pasadena CA, USA. LIGO Summer Undergraduate Research Fellow (SURF).}{Jun-Aug 2012}


\section{Education}

\cventry{2010 - 2013}{Master's degree in Astrophysics}{\newline Universit\`{a} degli Studi di Milano}{Milan, Italy}{}{}
\vspace{-0.1cm}
\begin{tabular}{rcl}
&\hspace{0.4cm} &$\circ\;\;${\textit{Final degree grade}}: 110/110 with distinction (``cum laude'').\\
&\hspace{0.4cm} &$\circ\;\;${\textit{Average class grade}}: 30/30 with 7/12 distinctions (``cum laude''). Top 1\% of my class.\\
&\hspace{0.4cm} &$\circ\;\;${\textit{Thesis advisors}}: Giuseppe Lodato, Emanuele Berti. Thesis resulted in 2 publications.\\
%\cvitem{}{Master's thesis: \emph{Black hole spin alignment in astrophysical environments}.}
\end{tabular}
 
 
\vspace{0.3cm}
\cventry{2007 - 2010}{Bachelor's degree in Physics}{Universit\`{a} degli Studi di Milano}{Milan, Italy}{}{}
\vspace{-0.1cm}
\begin{tabular}{rcl}
&\hspace{0.4cm} &$\circ\;\;${\textit{Final degree grade}}: 110/110 with distinction (``cum laude'').\\
&\hspace{0.4cm} &$\circ\;\;${\textit{Average class grade}}: 29.84/30 with 18/26 distinctions (``cum laude''). Top 1\% of my class.\\
%&\hspace{0.4cm} &$\circ\;\;${\textit{Thesis advisor}}:  Dietmar Klemm.\\
%\cvitem{}{Bachelor's thesis: \emph{Physical applications of Finsler geometry}.}
\end{tabular}
 
\vspace{0.3cm}
\cventry{2002 - 2007}{Scientific High School degree}{\newline Liceo Scientifico Statale Paolo Frisi}{Monza MB, Italy}{}{}  % 
\vspace{-0.1cm}
\begin{tabular}{rcl}
&\hspace{0.4cm} &$\circ\;\;${\textit{Final grade}}:  100/100.\\
\end{tabular}

\section{Grants, scholarships and awards}

\cvitemwithcomment{}{\textbf{Einstein Postdoctoral Prize Fellowship}, NASA \& Chandra X-ray center (Harvard).}{2016 - {current}} 

\cvitemwithcomment{}{\textbf{Isaac Newton Ph.D. Studentship}, DAMTP \& IoA, University of Cambridge.}{2013 - 2016} 

\cvitemwithcomment{}{\textbf{STFC Ph.D. Studentship}, UK Science \& Technology Facilities Council.}{2013 - 2016} 

\cvitemwithcomment{}{\textbf{Royal Astronomical Society Travel Grant}, Royal Astronomical Society.}{2016} 

\cvitemwithcomment{}{\textbf{Rouse Ball Travelling Studentship in Mathematics}, Trinity College, University of Cambridge.}{2015} 

\cvitemwithcomment{}{\textbf{Smith-Rayleigh-Knight Essay Prize}, Faculty of Mathematics, University of Cambridge.}{2015} 

\cvitemwithcomment{}{\textbf{LIGO Summer Undergraduate Research Fellowship}, California Institute of Technology.}{2012} 

\cvitemwithcomment{}{\textbf{Undergraduate Studentship}, Physics Department, University of Milan.}{2007 - 2008} 


%
%\cventry{2016 - {current}}{NASA \& Smithsonian Astrophysical Observatory (Harvard)}{Einstein Postdoctoral Prize Fellowship}{Pasadena CA, USA}{}{}
%
%\cventry{2013 - 2016}{DAMTP \& Institute of Astronomy}{Isaac Newton Ph.D. Studentship}{Cambridge, UK}{University of Cambridge}{}
%
%\vspace{0.05cm}
%\cventry{2013 - 2016}{UK Science \& Technology Facilities Council}{STFC Ph.D. Studentship}{Cambridge, UK}{}{}
%
%\vspace{0.05cm}
%\cventry{2016}{Royal Astronomical Society}{Royal Astronomical Society Travel Grant}{London, UK}{}{}
%
%\vspace{0.05cm}
%\cventry{2015}{Trinity College}{Rouse Ball Travelling Studentship in Mathematics}{Cambridge, UK}{University of Cambridge}{}
%
%\vspace{0.05cm}
%\cventry{2015}{Faculty of Mathematics}{Smith-Rayleigh-Knight Essay Prize}{Cambridge, UK}{University of Cambridge}{}

%\vspace{0.05cm}
%\cventry{2014-2015}{Darwin College}{Travel fund}{Cambridge, UK}{University of Cambridge}{}
%\vspace{0.05cm}
%\cventry{2015}{Cambridge Philosophical Society}{Travel fund}{Cambridge, UK}{University of Cambridge}{}
%
%\vspace{0.05cm}
%\cventry{2012}{LIGO Collaboration}{LIGO Summer Undergraduate Research Fellowship}{Pasadena CA, USA}{California Institute of Technology}{}
%
%\vspace{0.05cm}
%\cventry{2007 - 2008}{Physics Department, University of Milan}{Undergraduate Studentship}{Milan, Italy}{}{}
%
%\cvitem{Nov 2011}{Award assigned by the cultural association "Famiglia Legnanese" to  best Italian College students, Legnano MI, Italy.}
%\cvitem{Jul 2007}{Award assigned by the website \textit{matematicamente.it} to the best italian high-school research project, presenting my work  {"Do I Dare disturb the Universe?"} about  evidences for dark matter.}


\section{Publications}
%\vspace{0.2cm}

\cvitem{}{
\begin{tabular}{rcl}
\textbf{Counts}: &\hspace{0.3cm} &{\textbf{15} papers published in major peer-reviewed journals}, {\textbf{3} papers in submission stage,} \\
& &{(out of which \textbf{11} first-authored papers and \textbf{2} papers covered by press release),}
\\
& &{\textbf{2} other papers  published conference proceedings, software journals, etc.}
\end{tabular}
}
\textbf{Total number of citations:} 423 (using ADS and InSpire).

\textbf{h-index:} 11

%\textbf{Citations:} 342 citations in total (using ADS and InSpire); h-index 10

\textbf{Web links to list services:}
\href{http://labs.adsabs.harvard.edu/adsabs/search/?q=author%3A%22Gerosa%2C+Davide%22&month_from=&year_from=&month_to=&year_to=&db_f=&nr=&article=1&bigquery=&re_sort_type=CITED&re_sort_dir=desc}{\textsc{ADS}}; 
\href{http://inspirehep.net/search?ln=en&ln=en&p=exactauthor%3AD.Gerosa.1&of=hb&action_search=Search&sf=&so=d&rm=citation&rg=25&sc=0}{\textsc{InSpire}}; 
\href{http://arxiv.org/a/gerosa_d_1.html}{\textsc{arXiv}}.

\textbf{Full list of publications} available below and  at \href{http://www.tapir.caltech.edu/~dgerosa/pub}{\texttt{www.tapir.caltech.edu/\textasciitilde dgerosa/pub}}.



%%%%%%%%%%%%%%%%%%%%%%%%%%%%%%%%%
%%%%%%%%%%%%%%%%%%%%%%%%%%%%%%%%%
%%%%%%%%%%%%%%%%%%%%%%%%%%%%%%%%%



\vspace{+0.2cm}
\cvitem{\textcolor{color1}{Selected publications}}{}
\vspace{-0.7cm}

\cvitem{}{\small
\hspace{-1cm}\begin{longtable}{rp{0.3cm}p{17cm}}
\textbf{$\circ$} & & \textit{Black-hole kicks as new gravitational-wave observables.} 
\newline{}
{D.~Gerosa}, C. Moore.
  \newline{}
{\prl}~117 (2016) 011101.
arXiv:1606.04226 [gr-qc].  %\textbf{PRL Editors' Suggestion. Press release}.
\vspace{0.05cm}\\
\textbf{$\circ$} & & \textit{Precessional instability in binary black holes with aligned spins.} 
\newline{}
{D.~Gerosa}, M.~Kesden, R.~O’Shaughnessy, A.~Klein, E.~Berti, U.~Sperhake, D.~Trifir\`o.
  \newline{}
{\prl}~115 (2015) 141102.
arXiv:1506.09116 [gr-qc].  %\textbf{PRL Editors' Suggestion}.
\vspace{0.05cm}\\
\textbf{$\circ$} & & \textit{Resonant-plane locking and spin alignment in stellar-mass black-hole binaries: a diagnostic of compact-binary formation.}
\newline{}
{D.~Gerosa}, M.~Kesden, E.~Berti, R.~O'Shaughnessy, U.~Sperhake. 
\newline{}
\prd~87 (2013) 10, 104028. arXiv:1302.4442 [gr-qc].
\vspace{0.05cm}\\
%\textbf{$\circ$} & & \textit{Missing black holes in brightest cluster galaxies as evidence for the occurrence of superkicks in nature.}
%\newline{}
%\textbf{D.~Gerosa}, A.~Sesana.
%\newline{}
%\mnras~446 (2015) 38-55. arXiv:1405.2072 [astro-ph.GA]
%\vspace{0.05cm}\\
\end{longtable}
}


\section{Presentations}
%\vspace{0.3cm}

\cvitem{}{
%\begin{tabular}{rcl}
%\textbf{Presentation counts}: &\hspace{0.3cm} &{\textbf{11} talks at conferences} \\
%& &{\textbf{11} talks at department seminars}
% \\
%& &{\textbf{7} posters at conferences}
%\end{tabular}
\textbf{Counts}: {\textbf{16} talks at conferences}, {\textbf{14} talks at department seminars}, {\textbf{7} posters at conferences.}
%\end{tabular}
}

\textbf{Full list of presentations available} below and at \href{http://www.tapir.caltech.edu/~dgerosa/talks}{\texttt{www.tapir.caltech.edu/\textasciitilde dgerosa/talks}}.


\vspace{+0.2cm}
\cvitem{\textcolor{color1}{Selected presentations}}{}
\vspace{-0.7cm}

\cvitem{}{\small
\hspace{-1cm}\begin{longtable}{rp{0.3cm}p{17cm}}
\textbf{$\circ$} & & \textit{The kick is in the waveform: detection of black-hole recoils.}
\newline{}The Dawning Era of Gravitational-Wave Astrophysics, Aspen CO, USA, Feb 2017.
\vspace{0.05cm}\\
\textbf{$\circ$} & & \textit{Averaging the average: multi-timescale analysis of precessing black-hole binaries.}
\newline{}21st International Conference on General Relativity and Gravitation (GR21), New York NY, USA, Jul 2016.
\vspace{0.05cm}\\
\textbf{$\circ$} & & \textit{A new instability to black-hole spin precession.}
\newline{}28th Texas Symposium on Relativistic Astrophysics, Geneva, Switzerland, Dec 2015.
%\vspace{0.05cm}\\
%\textbf{$\circ$} & & \textit{Analytic solutions to binary black-hole spin precession: recalling Kepler's two-body problem.}
%\newline{}Compact Objects as Astrophysical and Gravitational Probes, Leiden, The Netherlands, Jan 2015.  \newline{} \textbf{Best young presentation award}
%\vspace{0.05cm}\\
%\textbf{$\circ$} & & \textit{Rival families: waveforms from resonant BH binaries as probes of their astrophysical formation history.}
%\newline{}3rd Session of the Sant Cugat Forum on Astrophysics, San Cugat, Spain, Apr 2014.  
\end{longtable}
}

\newpage{}

\section{Teaching and Mentoring}

\vspace{0.2cm}
%\textbf{\textcolor{black}{Ph.D. student projects:}}\vspace{0.05cm}\\ 
%\cvitemwithcomment{}{\hspace{0.4cm}$\circ\;$ R. Tso, Caltech.}{2017} \vspace{-0.1cm}
%\hspace{0.71cm} \textit{Improving black-hole spectroscopy with active interferometer techniques.}

\vspace{0.2cm}
\textbf{\textcolor{black}{Undergraduate student mentoring:}}\vspace{0.05cm}\\ 
\cvitemwithcomment{}{\hspace{0.4cm}$\circ\;$ J.~Vosmera, University of Cambridge.}{Summer 2015}\vspace{-0.1cm}
\hspace{0.71cm} \textit{On the equal-mass limit of precessing black-hole binaries} (published in CQG).

\cvitemwithcomment{}{\hspace{0.4cm}$\circ\;$ R.~Barbieri, University of Cambridge and University of Pavia.}
{Summer 2016} \vspace{-0.1cm}
\hspace{0.71cm} \textit{Measuring black-hole kicks with gravitational-wave observations}  (publication in prep.).

\cvitemwithcomment{}{\hspace{0.4cm}$\circ\;$ K.~Chamberlain, Caltech and Montana State University.}{Summer 2017, awarded} \vspace{-0.1cm}
\hspace{0.71cm} \textit{New frequency-domain gravitational-wave approximant to detect black hole recoils.}


%
\vspace{0.2cm}
\textbf{\textcolor{black}{Teaching assistant:}}\vspace{0.05cm}\\ 
\cvitemwithcomment{}{\hspace{0.4cm}$\circ\;$ University of Cambridge, Part III General Relativity (master class).}{2015-2016} \vspace{-0.1cm}
\cvitemwithcomment{}{\hspace{0.4cm}$\circ\;$  University of Cambridge, Part II General Relativity (3rd-year undergraduate class).}{2014-2016}


%\begin{tabular}{rcl}
%&\hspace{0.4cm} &$\circ\;\;$J.~Vosmera, University of Cambridge, Summer 2015\\
%&\hspace{0.4cm} &$\circ\;\;$R.~Barbieri, University of Cambridge and University of Pavia, Summer 2016\\

%J.~Vosmera, University of Cambridge, 2015;\

%&\hspace{0.4cm} &$\circ\;\;${\textbf{Institut Astrophysique de Paris}}. Paris, France. \textit{Jul 2015}\\
%&\hspace{0.4cm} &$\circ\;\;${\textbf{University of Mississippi}}. Oxford MS, USA. \textit{Aug-Dec 2012}\\
%&\hspace{0.4cm} &$\circ\;\;${\textbf{Caltech}}. Pasadena CA, USA. LIGO Summer Undergraduate Research Fellow. \textit{Jun-Aug 2012}\\

%\end{tabular}

%&\hspace{0.4cm} &$\circ\;\;${\textit{Teaching assistant}}: Introduction to General Relativity (Part II, 3rd year undergraduate class);\\
%&\hspace{0.4cm} &\hspace{3.5cm} Advanced General Relativity (Part III, master class).\\
%&\hspace{0.4cm} &$\circ\;\;${\textit{Undergraduate summer project mentoring}}: J.~Vosmera, University of Cambridge, 2015;\\
%&\hspace{0.4cm} &\hspace{7.3cm} R.~Barbieri, University of Pavia, 2016.\\
%\end{tabular}



%\cvitem{}{$\circ$ Developer of open-source Python module \textsc{precession}: \href{https://github.com/dgerosa/precession}{github.com/dgerosa/precession}. Complete toolbox to study the post-Newtonian dynamics of  spinning black-hole binaries (see Gerosa \& Kesden 2015).}
%\cvitem{Operating systems}{Mac OS, Linux, Windows.}
%\cvitem{Coding}{Python (advanced), Mathematica, C, Fortran, %(also with Root, RooFit and Numerical Recipes tools),\newline 
%Bash.}
%\cvitem{Other scientific tools}{LIGO lalsuite, LaTex, GIT, SVN, Gnuplot, SuperMongo.}


%\newpage{}

\section{Outreach and Service}

\textbf{\textcolor{black}{Referee}} for The Astrophysical Journal, Monthly Notices of the Royal Astronomical Society, General Relativity and Gravitation, The European Physical Journal Plus.

\vspace{0.2cm}
\textbf{\textcolor{black}{Code and data sharing}}\vspace{0.05cm}\\ 
\begin{tabular}{rcl}
&\hspace{0.4cm} &$\circ\;\;${\textit{Developer}} of open-source Python module \textsc{precession} (see \href{https://github.com/dgerosa/precession}{\texttt{github.com/dgerosa/precession}}). 
\\ &\hspace{0.4cm} &  \hspace{0.4cm}Toolbox to study the post-Newtonian dynamics of  spinning black-hole binaries; used in 10+  publications. \\ 
&\hspace{0.4cm} &$\circ\;\;${\textit{Open source project}}: \textsc{filltex}, automatic LaTex bibliography (see \href{https://github.com/dgerosa/filltex}{\texttt{github.com/dgerosa/filltex}}).\\
&\hspace{0.4cm} &$\circ\;\;${\textit{Research outputs}}: public catalog of numerical-relativity waveform in scalar-tensor theory of gravity.\\
\end{tabular}

\vspace{0.2cm}

\textbf{\textcolor{black}{Organized conferences}}\vspace{0.05cm}\\ 
\cvitemwithcomment{}{\hspace{0.4cm}$\circ\;$ \textit{The disc migration issue: from protoplanets to supermassive  black holes.}}{2017} \vspace{-0.1cm}
\cvitemwithcomment{}{\protect{$~\quad\;\;\;$}KAVLI Institute, University of Cambridge (SOC \& LOC).}{} \vspace{-0.1cm}
\cvitemwithcomment{}{\hspace{0.4cm}$\circ\;$ \textit{Einstein's Legacy: Celebrating 100 yrs of General Relativity.}}{2015} \vspace{-0.1cm}
\cvitemwithcomment{}{\protect{$~\quad\;\;\;$}Queen Mary University of London (SOC).}{} \vspace{-0.1cm}
%\cvitemwithcomment{}{\hspace{0.4cm}$\circ\;$ \textit{APS Pacific Coast Gravity Meeting.}}{2018, planned} \vspace{-0.1cm}
%\cvitemwithcomment{}{\protect{$~\quad\;\;\;$}California Institute of Technology (SOC \& LOC).}{} \vspace{-0.1cm}

\vspace{0.2cm}
\textbf{\textcolor{black}{Outreach activities}}\vspace{0.05cm}\\ 
\cvitemwithcomment{}{\hspace{0.4cm}$\circ\;$ TV Interview on gravitational waves and black-hole kicks aired on Cambridge TV.}{2016}\vspace{-0.1cm}
\cvitem{}{\hspace{0.8cm}(see \href{http://www.cambridge-tv.co.uk/davide-gerosa-black-holes/)}{\textit{cambridge-tv.co.uk/davide-gerosa-black-holes}})\vspace{-0.1cm}}
\cvitemwithcomment{}{\hspace{0.4cm}$\circ\;$ Actively involved in the Cambridge Science Festival, faculty of Mathematics.}{2015}\vspace{-0.1cm}
\cvitemwithcomment{}{\hspace{0.4cm}$\circ\;$ Research coverage by the Caltech outreach journal: CURJ, Vol.15 No.1 (2014).}{2014}\vspace{-0.1cm}
\cvitemwithcomment{}{\hspace{0.4cm}$\circ\;$ Scientific journalist for the Italian on-line newspaper \href{http://www.ilsussidiario.net}{\textit{ilsussidiario.net}.}}{since 2013}\vspace{-0.1cm}
\cvitemwithcomment{}{\hspace{0.4cm}$\circ\;$  Introductory astronomy and relativity lectures to high-school students.}{since 2011}\vspace{-0.1cm}


\vspace{0.2cm}
\textbf{\textcolor{black}{Academic Service}}\vspace{0.05cm}\\ 
\cvitemwithcomment{}{\hspace{0.4cm}$\circ\;$  PhD Representative, Gravitational Physics committee of the UK Institute of Physics.}{2014-2016}\vspace{-0.1cm}
\cvitemwithcomment{}{\hspace{0.4cm}$\circ\;$  Undergraduate Student Representative, Science Faculty Council, University of Milan, Italy.}{2008-2012}\vspace{-0.1cm}


\vspace{0.2cm}
\textbf{\textcolor{black}{Memberships}}: LISA Consortium, American Physical Society, Royal Astronomical Society.



%\cvitemwithcomment{}{Actively involved in the Cambridge Science Festival, faculty of Mathematics.}{2015}
%\cvitemwithcomment{}{Research coverage by the Caltech outreach journal: CURJ, Vol.15 No.1 (2014).}{2014}
%\cvitemwithcomment{}{Scientific journalist for the Italian on-line newspaper \href{http://www.ilsussidiario.net}{\textit{ilsussidiario.net}}}{2013-now}

%\cvitemwithcomment{}{Author in an undergraduate lecture-notes textbook on "Waves and Oscillations" (edited by CUSL press)}{2008}
%\mbox{(\url{euresis.org})}. 
%I contributed to the preparation of scientific outreach exhibitions presented at the \textit{Rimini Meeting}, a one-week cultural event visited by more than 800.000 people every year.}
%\vspace{-1\baselineskip}
%\cvitem{}{\small
%\begin{itemize}
%\item Is the atom really indivisible? Questions and certainties in science.
%\item From one to infinity. At the heart of mathematics.
%\item Things never seen before. Galileo, the struggle and wonder of a new gaze on the Universe.
%\item Atmosphera. Reality and myth of global changes
%\end{itemize}}
%\vspace{-1\baselineskip}
%
%
%\section{Extracurricular Activities}
%\cventry{2014-now}{PhD student representative}{Gravitational Physics Group, UK Institute of Physics}{London, UK}{}{}
%
%\cventry{2008-2012}{Elected undergraduate student representative}{Science Faculty Council and Physics Coordination Committee}{Milan, Italy}{University of Milan}{}
%
%\cventry{2004-2006}{High-school student representative, elected in public elections}{Regional High School Student Assembly}{Milan, Italy}{Milan city council, Italy}{}
%
%
%\section{Work Experiences}
%\cventry{2013}{Liceo Don Gnocchi}{High-school maths teacher}{Carate Brianza MB, Italy}{}{}
%\cventry{2013}{CEUR foundation}{Maths and physics tutoring for undergraduate students.}{Milan, Italy.}{}{}
%
%\cventry{2007-2012}{CUSL press}{Physical sciences publications manager.}{Milan, Italy.}{}{}{}
%
%\cvitemwithcomment{}{Introductory \emph{Astronomy} lectures to high-school classes}{{2011-2012}}%Liceo Linguistico M.Candia, Seregno MB, Italy; Collegio della Guastalla, Monza MB, Italy.}
%%\cvitem{2010}{IT Laboratory keeper; Department of Physics, Universit\`{a} degli Studi di Milano, Milan, Italy.}
%%\cvitem{2007 - 2013}{Long lasting experience in private tutoring sessions for high-school students (maths and physics).}
%\cvitemwithcomment{}{Private tutoring sessions for high-school students (maths and physics).}{2007-2013}
%
%%\cvitem{2007 - 2012}{Shop assistant and book orders, responsible for the publications in the Physics branch; CUSL press and bookstore, Milan, Italy.}
%


\section{Skills}
%\cvitem{Operating systems}{Mac OS, Linux, Windows.}
\cvitem{Programming languages}{Python (advanced), Bash, Mathematica, C, Fortran. %(also with Root, RooFit and Numerical Recipes tools),\newline 
}
\cvitem{Other scientific tools}{LIGO lalsuite, LaTex, GIT, SVN, Gnuplot, SuperMongo. Supercomputer jobs.}
\cvitem{Languages}{English (fluent), Italian (native).}


\section{Hobbies}
\cvitem{}{Alpinism, rock climbing, alpine skiing and ski touring.}
\cvitem{}{Soccer player (FC Inter fan). Jogging, tennis and squash as well.}
\cvitem{}{Rock music (great fan of Bruce Springsteen).}%{}{}{I reached the top of Monte Rosa (4554 meters, Italy) and Piz Palù (3882 meters, Switzerland)}{}


%CV_long
\end{document}
